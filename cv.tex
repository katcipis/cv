%%%%%%%%%%%%%%%%%%%%%%%%%%%%%%%%%%%%%%%%%
% Friggeri Resume/CV
% XeLaTeX Template
% Version 1.0 (5/5/13)
%
% This template has been downloaded from:
% http://www.LaTeXTemplates.com
%
% Original author:
% Adrien Friggeri (adrien@friggeri.net)
% https://github.com/afriggeri/CV
%
% License:
% CC BY-NC-SA 3.0 (http://creativecommons.org/licenses/by-nc-sa/3.0/)
%
% Important notes:
% This template needs to be compiled with XeLaTeX and the bibliography, if used,
% needs to be compiled with biber rather than bibtex.
%
%%%%%%%%%%%%%%%%%%%%%%%%%%%%%%%%%%%%%%%%%

\documentclass[]{friggeri-cv} % Add 'print' as an option into the square bracket to remove colors from this template for printing

\begin{document}

\header{tiago}{katcipis}{software engineer} % Your name and current job title/field

%----------------------------------------------------------------------------------------
%	SIDEBAR SECTION
%----------------------------------------------------------------------------------------

\begin{aside} % In the aside, each new line forces a line break
\section{contact}
Berlin,Germany
+39 349 859 0291
\href{mailto:tiagokatcipis@gmail.com}{tiagokatcipis@gmail.com}
\href{https://github.com/katcipis}{GitHub}
\href{http://www.linkedin.com/pub/tiago-katcipis/1b/273/8b0}{LinkedIn}
\href{http://katcipis.github.io/}{Blog}
\section{languages}
English, Portuguese
\section{programming languages}
Go, C, C++, Python, Lua, Bash, Javascript, Nash
\section{protocols}
HTTP, gRPC, SIP, RTP, AMQP
\section{cloud}
Kubernetes, AWS, Azure, Google Cloud, Docker, Terraform
\section{automation}
Ansible, Make
\section{dev environment}
Vagrant, Docker Compose
\section{monitoring}
Prometheus, Grafana, StatsD, Sysdig
\end{aside}

%----------------------------------------------------------------------------------------
%	INTRO SECTION
%----------------------------------------------------------------------------------------

\section{Introduction}

\begin{entrylist}
%------------------------------------------------
%------------------------------------------------
\entry
{}
{}
{}
{
I'm a curious programmer that likes to explore different ways to design, build and test software
always trying to understand as much as I can from the entire environment I'm working in.\\

That led me to do a lot of different things in my 10 years career, from embedded
software in C to distributed systems in a variety of languages and protocols.\\

I'm passionate about automation and minimalism when building scalable, efficient and flexible software.
}
%------------------------------------------------
\end{entrylist}

%----------------------------------------------------------------------------------------
%	WORK EXPERIENCE SECTION
%----------------------------------------------------------------------------------------

\section{experience}

\begin{entrylist}
%------------------------------------------------
\entry
{2019--present}
{FromAtoB}
{Berlin, Germany}
{\emph{Software Engineer - Search2Book Team} \\

\begin{itemize}
  \item Implemented new location service from scratch.
  \item Added PSD2 compliance on payment method storage service.
  \item Helped migrate core services from legacy environment to new GKE cluster.
  \item Built backup routines for critical service (Google Memorystore).
  \item Improved development environments making them more consistent.
  \item Integral part of the hiring process doing interviews and pair programming sessions.
\end{itemize}

}
\end{entrylist}

\begin{entrylist}
%------------------------------------------------
\entry
{2017--2019}
{Neoway}
{Florianópolis, Brazil}
{\emph{Software Engineer - Data Platform Team} \\

\begin{itemize}
  \item Led the migration of the entire data platform from AWS to Azure.
  \item Developed tools to automate building infrastructure, like {\href{https://github.com/NeowayLabs/klb}{klb}}.
  \item Created new service to solve audio captchas (Go, Python, SVM).
  \item Prototyped image captcha solver using TensorFlow.
\end{itemize}
}
\end{entrylist}

\begin{entrylist}
%------------------------------------------------
\entry
{2015--2017}
{Neoway}
{Florianópolis, Brazil}
{\emph{Lead Software Engineer - Data Capture Team} \\

\begin{itemize}
  \item Led development of a new data capture architecture.
  \item Implementation of multiple services for the new architecture (Python, Go).
  \item Added improved and fully automated monitoring system (Sysdig, StatsD).
  \item Coached the team on better testing practices and TDD.
  \item Fully automated dev environments and deployment (Docker, Docker Compose).
  \item First team on the company to deploy and use Kubernetes to manage more than 100 deployments.
\end{itemize}
}
\end{entrylist}

\begin{entrylist}
%------------------------------------------------
\entry
{2012--2015}
{Dígitro}
{Florianópolis, Brazil}
{\emph{Lead Software Engineer} \\

\begin{itemize}
  \item Developed VoIP phone with color touchscreen from scratch (C on a Blackfin DSP).
  \item Automated development environment for cross compilation (Ansible, Vagrant).
  \item Replaced legacy audio service that used Flash (RTMP) with an HTTP/HTML5 solution (NodeJS,C).
  \item Coached team on automated testing and TDD.
\end{itemize}
}
\end{entrylist}

\begin{entrylist}
%------------------------------------------------
\entry
{2010--2012}
{Dígitro}
{Florianópolis, Brazil}
{\emph{Software Engineer} \\

I started working on a solution to web audio playback
with very specific audio effects (like silence removal,
change in pitch) that had to be developed using Flash (RTMP).
To solve that problem I worked with
two different open source C++ projects that did reverse
engineering of the RTMP protocol to develop our own
Flash Media Server. I worked directly with the integration
of the server playback logic with Gstreamer and the plugins
that enabled the desired effects on playback. \\

The next project was a solution to biometric identification
using a third party C library that built and scored voice models.
I developed a REST service in Lua that integrated with C code
that built the voice models and used MongoDB to store the
voice models and perform searches on the database. \\
}
\end{entrylist}
%------------------------------------------------
\begin{entrylist}
\entry
{2008--2010}
{Dígitro}
{Florianópolis, Brazil}
{\emph{Trainee} \\

Helped in the development of an
cross platform (Windows and Linux) audio streaming
library for a VoIP softphone, aiming at porting
the current application that was Windows only to Linux. I also
got involved in the development of a prototype for a voice
biometrics system.
}
\end{entrylist}
%------------------------------------------------
\begin{entrylist}
\entry
{2007-2008}
{Cyclops / LAPIX}
{Florianópolis, Brazil}
{\emph{Trainee} \\

Worked on adding new features on the system responsible to
integrate medical equipment to the DICOM system, developing a
cross platform domain specific graphical XML editor.
This involved learning C++ and XML parsing, together with
developing cross platform GUI applications, on this case
using WxWidgets. The code has been tested using CppUnit.
}
\end{entrylist}
%------------------------------------------------

%----------------------------------------------------------------------------------------
%	PROJECTS SECTION
%----------------------------------------------------------------------------------------

\pagebreak
\section{open source projects}

\begin{entrylist}
\entry
{2017-now}
{mdtoc}
{\href{https://github.com/madlambda/mdtoc}{https://github.com/madlambda/mdtoc}}
{
A very simple table of contents generator for markdown.
}
%------------------------------------------------
\end{entrylist}

\begin{entrylist}
\entry
{2016-2018}
{nash}
{\href{https://github.com/NeowayLabs/nash}{https://github.com/NeowayLabs/nash}}
{
Nash is a shell language focused on simplicity and having a nicer syntax
than traditional shells and support to containers. It also strives to be
safer than traditional shells.
}
%------------------------------------------------
\end{entrylist}

\begin{entrylist}
\entry
{2016-2018}
{klb}
{\href{https://github.com/NeowayLabs/klb}{https://github.com/NeowayLabs/klb}}
{
klb is used to automate infrastructure creation on AWS and Azure.
I got involved on designing the support for Azure since this was
the tool used to migrate Neoway production infrastructure from
AWS to Azure.
}
%------------------------------------------------
\end{entrylist}

\begin{entrylist}
\entry
{2013}
{CppUTest}
{\href{http://cpputest.github.io}{http://cpputest.github.io}}
{
CppUTest is a C /C++ based unit xUnit test framework for unit
testing and for test-driving code.

In this project I worked both on improving the documentation and
at adding new native types to the mock framework (which involved
some refactoring).
}
%------------------------------------------------
\end{entrylist}

\begin{entrylist}
\entry
{2012}
{GStreamer}
{\href{http://www.gstreamer.net}{http://www.gstreamer.net}}
{
GStreamer is a library for constructing graphs of
media-handling components. I contributed with a plugin
named \emph{removesilence} and some documentation for the
GstCheck documentation.
}
%------------------------------------------------
\end{entrylist}

\begin{entrylist}
\entry
{2010-2011}
{Pattern detection on H.264}
{\href{https://github.com/katcipis/h264.pattern.detection}{https://github.com/katcipis/h264.pattern.detection}}
{
This is my Bachelor's Thesis and it consists of a prototype
of a H.264 CODEC that uses OpenCV and H.264 internal
algorithms to do pattern detection and
object tracking integrated on the encoding process.

Metadata generated on the encoding process is integrated on
the video bitstream on conformance with the standard.
}
%------------------------------------------------
\end{entrylist}

%------------------------------------------------

\begin{entrylist}
\entry
{2010-2011}
{LuaSofia}
{\href{https://github.com/ppizarro/luasofia}{https://github.com/ppizarro/luasofia}}
{
Lua binding for the Sofia-SIP library.
Contributed to the project from the start,
helping to make decisions about the design of the
software and documenting it.
}
%------------------------------------------------
\end{entrylist}

%------------------------------------------------
\begin{entrylist}
\entry
{2010}
{GPS tracking system}
{\href{https://github.com/katcipis/gps.tracking}{https://github.com/katcipis/gps.tracking}}
{
System designed to provide the location of a device
at the receive of a position request using SMS.
}
\end{entrylist}

%------------------------------------------------
\begin{entrylist}
\entry
{2010}
{LuaNotify}
{\href{https://github.com/katcipis/luanotify}{https://github.com/katcipis/luanotify}}
{
Lua library that implements a simple Pub/Sub system
inspired on glib GSignal API.
}
%------------------------------------------------
\end{entrylist}
\pagebreak

\section{presentations}

%------------------------------------------------
\begin{entrylist}
\entry
{2018}
{Object Orientation in Go}
{\href{http://www.thedevelopersconference.com.br/tdc/2018/florianopolis/trilha-golang}{The Developers Conference}}
{

For people that come from a background on Java or other classic object oriented languages
(like C++) there is also some discussion on if Go is actually object oriented.

In this presentation I try to present Go as a language that is more object oriented than these
classic languages, at least according to the original foundations of object orientation.

Presentation source can be found \href{https://github.com/katcipis/my.presentations/blob/master/gooo/gooo.slide}{here}.

}
%------------------------------------------------
\end{entrylist}

%------------------------------------------------
\begin{entrylist}
\entry
{2016}
{Building Resilient Services in Go}
{\href{https://2016.gopherconbr.org/en/}{GopherCon Brazil}}
{

Resilience is not about never failing, but how do you recover from it.
How can you prevent your services from locking down or exhausting all its resources ?
How to perform graceful service degradation ? Can this kind of behaviour be tested properly ?\\

On Go we have some new features, like Contexts, that helps us to model timeouts and cancellation properly.\\

They can be combined with other useful features as select and channels to model timeouts and resource pools,
which can be essential to provide proper service degradation instead of total failure of the system.\\

On this talk I try to answer this questions using new features available on Go 1.7, direct from production ready software.\\

Presentation source can be found \href{https://github.com/katcipis/my.presentations/tree/master/resilient-services-in-go}{here}.

}
%------------------------------------------------
\end{entrylist}

%------------------------------------------------
\begin{entrylist}
\entry
{2016}
{Real Life Kubernetes}
{\href{http://www.thedevelopersconference.com.br/tdc/2016/florianopolis/trilha-devops}{The Developers Conference}}
{

On this presentation we will give a short introduction on Kubernetes and show the experience of learning
and using Kubernetes on production for two very distinct systems.\\

The first one is a data acquisition system, involving running multiple instances of different crawlers,
storage services, captcha breaking services, message brokers (like RabbitMQ) and database integration outside the cluster.\\

The second one is a web application, involving network analysis using graphs with the ultimate goal of fraud prevention.
The application is strongly bounded with the microservices architecture and the twelve factor app.\\

Graph and document databases, cache layers, a message broker and a distributed filesystem are some of
the technologies surrounding the application ecosystem.\\

Presentation source can be found \href{https://github.com/katcipis/my.presentations/tree/master/real-life-kubernetes}{here}.

}
%------------------------------------------------
\end{entrylist}

\end{document}
