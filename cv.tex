%%%%%%%%%%%%%%%%%%%%%%%%%%%%%%%%%%%%%%%%%
% Friggeri Resume/CV
% XeLaTeX Template
% Version 1.0 (5/5/13)
%
% This template has been downloaded from:
% http://www.LaTeXTemplates.com
%
% Original author:
% Adrien Friggeri (adrien@friggeri.net)
% https://github.com/afriggeri/CV
%
% License:
% CC BY-NC-SA 3.0 (http://creativecommons.org/licenses/by-nc-sa/3.0/)
%
% Important notes:
% This template needs to be compiled with XeLaTeX and the bibliography, if used,
% needs to be compiled with biber rather than bibtex.
%
%%%%%%%%%%%%%%%%%%%%%%%%%%%%%%%%%%%%%%%%%

\documentclass[]{friggeri-cv} % Add 'print' as an option into the square bracket to remove colors from this template for printing

\begin{document}

\header{tiago}{katcipis}{software engineer} % Your name and current job title/field

%----------------------------------------------------------------------------------------
%	SIDEBAR SECTION
%----------------------------------------------------------------------------------------

\begin{aside} % In the aside, each new line forces a line break
\section{contact}
Berlin,Germany
+39 349 859 0291
\href{mailto:tiagokatcipis@gmail.com}{tiagokatcipis@gmail.com}
\href{https://github.com/katcipis}{GitHub}
\href{http://www.linkedin.com/pub/tiago-katcipis/1b/273/8b0}{LinkedIn}
\href{http://katcipis.github.io/}{Blog}
\section{languages}
English, Portuguese
\section{programming languages}
Go, C, C++, Python, Lua, Bash, Javascript, Nash
\section{protocols}
HTTP, gRPC, SIP, RTP, AMQP
\section{cloud}
Kubernetes, AWS, Azure, Google Cloud, Docker, Terraform
\section{automation}
Ansible, Make
\section{dev environment}
Vagrant, Docker Compose
\section{monitoring}
Prometheus, Grafana, StatsD, Sysdig
\end{aside}

%----------------------------------------------------------------------------------------
%	INTRO SECTION
%----------------------------------------------------------------------------------------

\section{Introduction}

\begin{entrylist}
%------------------------------------------------
%------------------------------------------------
\entry
{}
{}
{}
{
I'm a curious programmer that likes to explore different ways to design, build and test software
always trying to understand as much as I can from the entire environment I'm working in.\\

That led me to do a lot of different things in my 10 years career, from embedded
software in C to distributed systems in a variety of languages and protocols.\\

I'm passionate about automation and minimalism when building scalable, efficient and flexible software.
}
%------------------------------------------------
\end{entrylist}

%----------------------------------------------------------------------------------------
%	WORK EXPERIENCE SECTION
%----------------------------------------------------------------------------------------

\section{experience}

\begin{entrylist}
%------------------------------------------------
\entry
{2019--present}
{FromAtoB}
{Berlin, Germany}
{\emph{Software Engineer - Search2Book Team} \\

\begin{itemize}
  \item Implemented new location service from scratch.
  \item Added PSD2 compliance on payment method storage service.
  \item Helped migrate core services from legacy environment to new GKE cluster.
  \item Built backup routines for critical service (Google Memorystore).
  \item Improved development environments making them more consistent.
  \item Integral part of the hiring process doing interviews and pair programming sessions.
\end{itemize}

}
\end{entrylist}

\begin{entrylist}
%------------------------------------------------
\entry
{2017--2019}
{Neoway}
{Florianópolis, Brazil}
{\emph{Software Engineer - Data Platform Team} \\

\begin{itemize}
  \item Led the migration of the entire data platform from AWS to Azure.
  \item Developed tools to automate building infrastructure, like {\href{https://github.com/NeowayLabs/klb}{klb}}.
  \item Created new service to solve audio captchas (Go, Python, SVM).
  \item Prototyped image captcha solver using TensorFlow.
  \item Did 3 different presentations in 2 different conferences.
\end{itemize}
}
\end{entrylist}

\begin{entrylist}
%------------------------------------------------
\entry
{2015--2017}
{Neoway}
{Florianópolis, Brazil}
{\emph{Lead Software Engineer - Data Capture Team} \\

\begin{itemize}
  \item Led development of a new data capture architecture.
  \item Implementation of multiple services for the new architecture (Python, Go).
  \item Added improved and fully automated monitoring system (Sysdig, StatsD).
  \item Coached the team on better testing practices and TDD.
  \item Fully automated dev environments and deployment (Docker, Docker Compose).
  \item First team on the company to deploy and use Kubernetes to manage more than 100 deployments.
\end{itemize}
}
\end{entrylist}

\begin{entrylist}
%------------------------------------------------
\entry
{2012--2015}
{Dígitro}
{Florianópolis, Brazil}
{\emph{Lead Software Engineer} \\

\begin{itemize}
  \item Developed VoIP phone with color touchscreen from scratch (C on a Blackfin DSP).
  \item Automated development environment for cross compilation (Ansible, Vagrant).
  \item Replaced legacy audio service that used Flash (RTMP) with an HTTP/HTML5 solution (NodeJS,C).
  \item Coached team on automated testing and TDD.
\end{itemize}
}
\end{entrylist}

\begin{entrylist}
%------------------------------------------------
\entry
{2010--2012}
{Dígitro}
{Florianópolis, Brazil}
{\emph{Software Engineer} \\

I started working on a solution to web audio playback
with very specific audio effects (like silence removal,
change in pitch) that had to be developed using Flash (RTMP).
To solve that problem I worked with
two different open source C++ projects that did reverse
engineering of the RTMP protocol to develop our own
Flash Media Server. I worked directly with the integration
of the server playback logic with Gstreamer and the plugins
that enabled the desired effects on playback. \\

The next project was a solution to biometric identification
using a third party C library that built and scored voice models.
I developed a REST service in Lua that integrated with C code
that built the voice models and used MongoDB to store the
voice models and perform searches on the database. \\
}
\end{entrylist}
%------------------------------------------------
\begin{entrylist}
\entry
{2008--2010}
{Dígitro}
{Florianópolis, Brazil}
{\emph{Trainee} \\

Helped in the development of an
cross platform (Windows and Linux) audio streaming
library for a VoIP softphone, aiming at porting
the current application that was Windows only to Linux. I also
got involved in the development of a prototype for a voice
biometrics system.
}
\end{entrylist}
%------------------------------------------------
\begin{entrylist}
\entry
{2007-2008}
{Cyclops / LAPIX}
{Florianópolis, Brazil}
{\emph{Trainee} \\

Worked on adding new features on the system responsible to
integrate medical equipment to the DICOM system, developing a
cross platform domain specific graphical XML editor.
This involved learning C++ and XML parsing, together with
developing cross platform GUI applications, on this case
using WxWidgets. The code has been tested using CppUnit.
}
\end{entrylist}
%------------------------------------------------

\end{document}
