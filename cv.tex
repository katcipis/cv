%%%%%%%%%%%%%%%%%%%%%%%%%%%%%%%%%%%%%%%%%
% Friggeri Resume/CV
% XeLaTeX Template
% Version 1.0 (5/5/13)
%
% This template has been downloaded from:
% http://www.LaTeXTemplates.com
%
% Original author:
% Adrien Friggeri (adrien@friggeri.net)
% https://github.com/afriggeri/CV
%
% License:
% CC BY-NC-SA 3.0 (http://creativecommons.org/licenses/by-nc-sa/3.0/)
%
% Important notes:
% This template needs to be compiled with XeLaTeX and the bibliography, if used,
% needs to be compiled with biber rather than bibtex.
%
%%%%%%%%%%%%%%%%%%%%%%%%%%%%%%%%%%%%%%%%%

\documentclass[]{friggeri-cv} % Add 'print' as an option into the square bracket to remove colors from this template for printing

\begin{document}

\header{tiago}{katcipis}{software developer} % Your name and current job title/field

%----------------------------------------------------------------------------------------
%	SIDEBAR SECTION
%----------------------------------------------------------------------------------------

\begin{aside} % In the aside, each new line forces a line break
\section{contact}
+(55)(48)991060132
~
\href{mailto:tiagokatcipis@gmail.com}{tiagokatcipis@gmail.com}
\href{https://github.com/katcipis}{GitHub}
\href{http://www.linkedin.com/pub/tiago-katcipis/1b/273/8b0}{LinkedIn}
\href{http://katcipis.github.io/}{Blog}
\section{languages skill}
English: advanced (reading, writing); intermediate (speaking)
\section{programming languages}
C, C++, Go, Python, Lua, Nash, Javascript, Bash, Java
\section{protocols}
DBus, RTMP, HTTP, SMTP, SIP, RTP, AMQP
\section{cloud}
Kubernetes, AWS, Azure, Docker, CoreOS, Terraform
\section{documentation}
Gtk-Doc, Doxygen, Docbook, LDoc, LaTeX, Docco
\section{automation}
Ansible, Make, Grunt
\section{dev environment}
Vagrant, Docker Compose
\section{monitoring}
Prometheus, Grafana, StatsD, Sysdig
\end{aside}

%----------------------------------------------------------------------------------------
%	EDUCATION SECTION
%----------------------------------------------------------------------------------------

\section{education}

\begin{entrylist}
%------------------------------------------------
%------------------------------------------------
\entry
{2006--2011}
{Bachelor of Computer Science}
{UFSC}
{}
%------------------------------------------------
\end{entrylist}

%----------------------------------------------------------------------------------------
%	WORK EXPERIENCE SECTION
%----------------------------------------------------------------------------------------

\section{experience}

\begin{entrylist}
%------------------------------------------------
\entry
{2018--now}
{Neoway}
{Florianópolis, Santa Catarina}
{\emph{Software Developer - Lambda Team} \\

    As a software developer on the lambda team I was responsible for researching
    and developing new captcha solvers (audio and image). Solving audio captchas involved
    studying and applying the following algorithms and tools:

\begin{itemize}
    \item Noise removal using sox
    \item Audio segmentation by silence using sox
    \item Extracting features (MFCC) from segmented audio
    \item Normalize extracted features with rescaling
    \item Train SVM model
    \item Recognize characters using trained SVM model
\end{itemize}

Main accomplishments:\\

\begin{itemize}
    \item Development of tool to measure assertiveness of captcha solving services
    \item Solved audio captcha with 80\% of assertiveness
\end{itemize}


}
\end{entrylist}


\begin{entrylist}
%------------------------------------------------
\entry
{2017--2018}
{Neoway}
{Florianópolis, Santa Catarina}
{\emph{Software Developer - Data Platform Team} \\

    As a software developer on the data platform team I started to be
    responsible for a more broad context involving multiple teams, building not
    only services but also tooling to be used by those teams. \\
    Besides the technical challenges of providing a platform I also performed
    code reviews and coaching across multiple teams.
    There was a lot of challenges like:\\

\begin{itemize}
    \item Migrating the entire infrastructure of multiple teams from AWS to Azure
    \item Defining network topology and security measures
    \item Creating backups for VMs with different operational systems at Azure
    \item Specify a new DSL to develop web spiders
    \item Scale the ingestion of huge CSV files (terabytes per file)
    \item Define solutions for infrastructure building that crossed multiple teams boundaries
    \item Performance problems on Postgres at Azure
\end{itemize}

Main accomplishments:

\begin{itemize}
    \item Fully automated infrastructure provisioning at Azure
    \item Easy to build parallel version of the entire infrastructure (including networking and security)
    \item Development and testing of a library to automate infrastructure building
    \item Specified and developed generic backup strategy at Azure
    \item Entire infrastructure access through a bastion host as a security measure
    \item New service that allowed arbitrarily huge files to be ingested easily (scaling horizontaly)
\end{itemize}
}
\end{entrylist}

\begin{entrylist}
%------------------------------------------------
\entry
{2015--2017}
{Neoway}
{Florianópolis, Santa Catarina}
{\emph{Software Developer - Data Capture Team} \\

    Technical lead of the team responsible for capturing data from the web and publishing it to the entire company,
which involved solving hard problems as:\\

\begin{itemize}
    \item Scraping the web (Scrapy and Selenium)
    \item Parsing data from multiple formats (HTML, PDF, SWF)
    \item Maintaining all downloaded raw data at S3
    \item Defining and documenting good protocols for proper service integration
    \item Developing services on different languages, like Go and Python
    \item Migrating the whole data pipeline to a new service oriented architecture
    \item Coach the team on good test techniques for the new architecture
    \item Developed new services to provide good proxies and captcha breaking
\end{itemize}

Besides the technical challenges I also helped the team to apply some
development practices, like TDD (Test Driven Development),code review
and continuous integration.
Together we built a DevOps culture to enable infrastructure as code
on our team, we where responsible for the whole solution, from the development
to testing and deployment (including monitoring the production system).\\

Main accomplishments:

\begin{itemize}
    \item Technical leadership and coaching for a team of 6 people
    \item Built a new crawling framework, from scratch to production
    \item Fully automated development environment with docker compose
    \item Fully automated dashboard and alert creation on Sysdig Cloud
    \item Cluster orchestration at AWS/Azure using Kubernetes, CoreOS and Docker
    \item Migrating all infrastructure from Codero to AWS
    \item Later, migrated the infrastructure from AWS to Azure
    \item Active participation on building the tools to automate infrastructure on Azure
    \item Aided on the migration of more than 100 crawlers to Kubernetes
    \item Actively participated on the screening and interviews of new candidates for the team
    \item Coached new members
    \item Implemented a real time monitoring for the data production pipeline using StatsD and Sysdig
    \item Gave talks inside the company and on events like TDC (The Developers Conference) and GopherCon
\end{itemize}
}
\end{entrylist}

\begin{entrylist}
%------------------------------------------------
\entry
{2012--2015}
{Dígitro}
{Florianópolis, Santa Catarina}
{\emph{Systems Analyst} \\

Technical lead of the team, helping on design, implementation and testing of new and legacy software.\\

Development of a REST service responsible for audio streaming and audio metadata extraction.\\

Helping the team to apply agile development practices, like TDD, self organization and continuous integration.\\

Evangelizing the adoption of the DevOps culture to enable infrastructure as code on the entire organization.\\

Helped on the adoption of new technologies like NodeJS, defining good practices and the tool set.\\

Coaching new members on the team.\\

Main accomplishments:\\

\begin{itemize}
\item Defining the architecture and contract of a REST service from scratch.
\item Implementing a REST service with TDD.
\item Acquired knowledge with NodeJS, Express, Mocha, Istanbul, JSHint, Grunt.
\item Integration of a NodeJS server with multiple child processes.
\item Developing GStreamer plugins with TDD.
\item Constructed development environments with Vagrant.
\item Orchestration with Ansible.
\item Performed presentations to disseminate the idea of distributed/automated development environments.
\item Coaching the use of TDD on the team.
\item Helping the team to introduce automated tests on legacy code.
\item Coached two new members on the team.
\item Usage of CRC cards to design systems.
\item Contributing on the migration process from svn to git.
\item Experience integrating heterogeneous systems.
\end{itemize}
}
\end{entrylist}

\begin{entrylist}
%------------------------------------------------
\entry
{2010--2012}
{Dígitro}
{Florianópolis, Santa Catarina}
{\emph{Programmer} \\

Development of a VoIP phone microservices, participating actively on the design and
architecture of the solutions.\\

Some of the services include:\\
\begin{itemize}
\item Audio streaming with RTMP.
\item Voice biometrics.
\item Word searching on audio.
\item REST interface for a PBX.\\
\end{itemize}

Main accomplishments:\\

\begin{itemize}
\item Acquired knowledge on interprocess communication using DBus.
\item Object orientation in C.
\item TDD on embedded C.
\item Development of VoIP applications.
\item Embedding Lua code on C.
\item Defining the API and architecture of web services and middlewares.
\item Embedded developing on blackfin platform, using uCLinux and UBoot.
\item Continuous integration using Jenkins.
\item Experience with document oriented databases.
\item Database replication with MongoDB.\\
\end{itemize}
}
\end{entrylist}
%------------------------------------------------
\begin{entrylist}
\entry
{2008--2010}
{Dígitro}
{Florianópolis, Santa Catarina}
{\emph{Trainee} \\

Developing cross platform applications (Windows and Linux) for audio streaming.
Development of a prototype for a voice biometrics system.

Main accomplishments:\\

\begin{itemize}
\item Cross compiling code to Windows using mingw.
\item Developing a cross platform native application in Python.
\item Developing a simple web server in Python.\\
\end{itemize}

}
\end{entrylist}
%------------------------------------------------
\begin{entrylist}
\entry
{2007-2008}
{Cyclops / LAPIX}
{Florianópolis, Santa Catarina}
{\emph{Trainee} \\

Developing new features on the DICOMizer, a system responsible to integrate medical equipment to the DICOM system.

Main accomplishments:\\

\begin{itemize}
\item Experience with C++ development and XML parsing.
\item Developing a native frontend with C++ and wXwidgets.
\item Automating the development environment with Python.
\item Test automation using CppUnit.
\end{itemize}
}
\end{entrylist}
%------------------------------------------------

%----------------------------------------------------------------------------------------
%	PROJECTS SECTION
%----------------------------------------------------------------------------------------

\pagebreak
\section{open source projects}

\begin{entrylist}
\entry
{2016-now}
{Nash}
{\href{https://github.com/NeowayLabs/nash}{https://github.com/NeowayLabs/nash}}
{
Nash is a shell language focused on simplicity and having a nicer syntax
than traditional shells. I use this shell as my interactive shell and it is
a pet project very close to my heart.
}
%------------------------------------------------
\end{entrylist}

\begin{entrylist}
\entry
{2016-now}
{klb}
{\href{https://github.com/NeowayLabs/klb}{https://github.com/NeowayLabs/klb}}
{
    klb is used to automate infrastructure creation on AWS and Azure.
    My development involvement is more focused on Azure, specially when
    we migrated Neoway infrastructure from AWS to Azure.
}
%------------------------------------------------
\end{entrylist}

\begin{entrylist}
\entry
{2013}
{CppUTest}
{\href{http://cpputest.github.io}{http://cpputest.github.io}}
{
CppUTest is a C /C++ based unit xUnit test framework for unit testing and for test-driving code.

Main accomplishments:\\
\begin{itemize}
\item Improving documentation.
\item Implementing new features on the mocking infrastructure.
\item Refactoring on the mocking infrastructure.
\end{itemize}
}
%------------------------------------------------
\end{entrylist}

\begin{entrylist}
\entry
{2012}
{GStreamer}
{\href{http://www.gstreamer.net}{http://www.gstreamer.net}}
{
GStreamer is a library for constructing graphs of media-handling components.

Main accomplishments:\\
\begin{itemize}
\item Developing the \emph{removesilence} plugin.
\item Improving GstCheck documentation.
\end{itemize}
}
%------------------------------------------------
\end{entrylist}

\begin{entrylist}
\entry
{2010-2011}
{Pattern detection on H.264}
{\href{https://github.com/katcipis/h264.pattern.detection}{https://github.com/katcipis/h264.pattern.detection}}
{
This project is a prototype of a H.264 CODEC that uses OpenCV and H.264 internal algorithms to do pattern detection and
object tracking integrated on the encoding process.

Metadata generated on the encoding process is integrated on the video bitstream on conformance with the standard.

Main accomplishments:\\
\begin{itemize}
\item Development of research abilities and textual elaboration of the results obtained.
\item Usage of a machine learning algorithm (Haar) to detect patterns.
\item Object oriented C code integrated to a large C code base (the H.264 reference CODEC).
\item Understanding of the motion estimation algorithms present on H.264.
\item Writing documents in LaTeX.
\end{itemize}
}
%------------------------------------------------
\end{entrylist}

\begin{entrylist}
%------------------------------------------------
\entry
{2011}
{Smartgrid - Access Control}
{\href{https://github.com/katcipis/smartbuilding.accesscontrol}{https://github.com/katcipis/smartbuilding.accesscontrol}}
{

Access control system designed to work on a smartgrid. Consists on tablets controlling access to the rooms based on
a authentication server where the identification is made using RFID.

Also developed the server where new users are registered.

Main accomplishments on the server:\\
\begin{itemize}
\item Developing Python server using the ICE protocol (Internet Communications Engine).
\item Document oriented database using CouchDB.
\item Native GUI using QT (PySide).
\item Integration with a RFID USB reader.
\end{itemize}

Main accomplishments on the client:\\
\begin{itemize}
\item Developing on Android 3.0, integrating Java and C++ code (using NDK and JNI).
\item Communication with a server using the ICE protocol.
\end{itemize}
}
\end{entrylist}
%------------------------------------------------

\begin{entrylist}
\entry
{2010-2011}
{LuaSofia}
{\href{https://github.com/ppizarro/luasofia}{https://github.com/ppizarro/luasofia}}
{

Lua binding for the Sofia-SIP library.
Contributed to the project from the start, helping to make decisions about the design of the software and documenting it.

Main accomplishments:\\
\begin{itemize}
\item Binding Lua to C.
\item Understanding of the SIP protocol.
\end{itemize}
}
%------------------------------------------------
\end{entrylist}

%------------------------------------------------
\begin{entrylist}
\entry
{2010}
{GPS tracking system}
{\href{https://github.com/katcipis/gps.tracking}{https://github.com/katcipis/gps.tracking}}
{

System designed to provide the location of a device through SMS. 

Main accomplishments:\\
\begin{itemize}
\item Integrating an AVR microcontroller with a G.24 Motorola modem and a GPS.
\item C++ code optimized to embedded systems (no heap allocations or kernel involved).
\item Understanding of the Motorola G.24 AT protocol.
\end{itemize}
}
\end{entrylist}

%------------------------------------------------
\begin{entrylist}
\entry
{2010}
{LuaNotify}
{\href{https://github.com/katcipis/luanotify}{https://github.com/katcipis/luanotify}}
{

Lua library that implements the Pub/Sub pattern, inspired on glib GSignal.

Main accomplishments:\\
\begin{itemize}
\item Developing Lua code with TDD.
\item Documenting Lua code with LDoc.
\item Packaging Lua code with LuaRocks.
\end{itemize}
}
%------------------------------------------------
\end{entrylist}
\pagebreak

\section{presentations}

%------------------------------------------------
\begin{entrylist}
\entry
{2018}
{Object Orientation in Go}
{\href{http://www.thedevelopersconference.com.br/tdc/2018/florianopolis/trilha-golang}{The Developers Conference}}
{

For people that come from a background on Java or other classic object oriented languages
(like C++) there is also some discussion on if Go is actually object oriented.

In this presentation I try to present Go as a language that is more object oriented than these
classic languages, at least according to the original foundations of object orientation.

Presentation source can be found \href{https://github.com/katcipis/my.presentations/blob/master/gooo/gooo.slide}{here}.

}
%------------------------------------------------
\end{entrylist}

%------------------------------------------------
\begin{entrylist}
\entry
{2016}
{Building Resilient Services in Go}
{\href{https://2016.gopherconbr.org/en/}{GopherCon Brazil}}
{

Resilience is not about never failing, but how do you recover from it.
How can you prevent your services from locking down or exhausting all its resources ?
How to perform graceful service degradation ? Can this kind of behaviour be tested properly ?\\

On Go we have some new features, like Contexts, that helps us to model timeouts and cancellation properly.\\

They can be combined with other useful features as select and channels to model timeouts and resource pools,
which can be essential to provide proper service degradation instead of total failure of the system.\\

On this talk I try to answer this questions using new features available on Go 1.7, direct from production ready software.\\

Presentation source can be found \href{https://github.com/katcipis/my.presentations/tree/master/resilient-services-in-go}{here}.

}
%------------------------------------------------
\end{entrylist}

%------------------------------------------------
\begin{entrylist}
\entry
{2016}
{Real Life Kubernetes}
{\href{http://www.thedevelopersconference.com.br/tdc/2016/florianopolis/trilha-devops}{The Developers Conference}}
{

On this presentation we will give a short introduction on Kubernetes and show the experience of learning
and using Kubernetes on production for two very distinct systems.\\

The first one is a data acquisition system, involving running multiple instances of different crawlers,
storage services, captcha breaking services, message brokers (like RabbitMQ) and database integration outside the cluster.\\

The second one is a web application, involving network analysis using graphs with the ultimate goal of fraud prevention.
The application is strongly bounded with the microservices architecture and the twelve factor app.\\

Graph and document databases, cache layers, a message broker and a distributed filesystem are some of
the technologies surrounding the application ecosystem.\\

Also making a great case of integration between AWS and k8s using tools such as autoscaling.\\

Presentation source can be found \href{https://github.com/katcipis/my.presentations/tree/master/real-life-kubernetes}{here}.

}
%------------------------------------------------
\end{entrylist}

\pagebreak

%----------------------------------------------------------------------------------------
%	INTERESTS SECTION
%----------------------------------------------------------------------------------------

\section{interests}

\textbf{professional:} distributed systems, cloud computing, test driven development, software engineering, software architecture, networks, plan9, linux, open source, artificial intelligence, image processing, dsp, functional programming, embedded systems

\textbf{books:} The Idea Factory, The Practice of Programming, Hackers: Heroes Of The Computer Revolution, The Pragmatic Programmer, Pragmatic Thinking \& Learning, Test Driven Development for Embedded C, REST in Practice, Refactoring: Improving the Design of Existing Code.

\end{document}
