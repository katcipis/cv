%%%%%%%%%%%%%%%%%%%%%%%%%%%%%%%%%%%%%%%%%
% Friggeri Resume/CV
% XeLaTeX Template
% Version 1.0 (5/5/13)
%
% This template has been downloaded from:
% http://www.LaTeXTemplates.com
%
% Original author:
% Adrien Friggeri (adrien@friggeri.net)
% https://github.com/afriggeri/CV
%
% License:
% CC BY-NC-SA 3.0 (http://creativecommons.org/licenses/by-nc-sa/3.0/)
%
% Important notes:
% This template needs to be compiled with XeLaTeX and the bibliography, if used,
% needs to be compiled with biber rather than bibtex.
%
%%%%%%%%%%%%%%%%%%%%%%%%%%%%%%%%%%%%%%%%%

\documentclass[]{friggeri-cv} % Add 'print' as an option into the square bracket to remove colors from this template for printing

\begin{document}

\header{tiago}{katcipis}{software engineer} % Your name and current job title/field

%----------------------------------------------------------------------------------------
%	SIDEBAR SECTION
%----------------------------------------------------------------------------------------

\begin{aside} % In the aside, each new line forces a line break
\section{contact}
+383498590291
\href{mailto:tiagokatcipis@gmail.com}{tiagokatcipis@gmail.com}
\href{https://github.com/katcipis}{GitHub}
\href{http://www.linkedin.com/pub/tiago-katcipis/1b/273/8b0}{LinkedIn}
\href{http://katcipis.github.io/}{Blog}
\section{languages skill}
English: advanced
\section{programming languages}
Go, C, C++, Python, Lua, Nash, Bash, Javascript
\section{protocols}
DBus, RTMP, HTTP, SIP, RTP, AMQP
\section{cloud}
Kubernetes, AWS, Azure, Google Cloud, Docker, CoreOS, Terraform
\section{automation}
Ansible, Make
\section{dev environment}
Vagrant, Docker Compose
\section{monitoring}
Prometheus, Grafana, StatsD, Sysdig
\end{aside}

%----------------------------------------------------------------------------------------
%	EDUCATION SECTION
%----------------------------------------------------------------------------------------

\section{education}

\begin{entrylist}
%------------------------------------------------
%------------------------------------------------
\entry
{2006--2011}
{Bachelor of Computer Science}
{UFSC}
{}
%------------------------------------------------
\end{entrylist}

%----------------------------------------------------------------------------------------
%	WORK EXPERIENCE SECTION
%----------------------------------------------------------------------------------------

\section{experience}

\begin{entrylist}
%------------------------------------------------
\entry
{2017--now}
{Neoway}
{Florianópolis, Santa Catarina}
{\emph{Software Engineer - Data Platform Team} \\

The first problem to solve was the migration
of the entire data platform (capture, transformation
and storage of data) from AWS to Azure. The new platform
at Azure needed to be fully automated to the point
that building the entire platform required only one
command and enabled the automated build
of staging environments.\\

We also needed to improve security
which was done using a bastion server. All this was developed
using an open source project developed at Neoway to deploy
infrastructure at Azure called
{\href{https://github.com/NeowayLabs/klb}{klb}} which is
developed in Nash. The final
tooling isolated the user from all this choices using docker
and provided a CLI. In the end we were able to build
multiple copies of the entire infrastructure (with different
sizing) and this was used by multiple teams to deploy
their infrastructure at Azure.\\

In the sequence we had some problems with captchas that
required us to solve audio captchas, something that had not
been done before at Neoway (we had infrastructure for image ones).
Doing that involved some researching
on current techniques and culminated on a service capable
of solving a specific audio captcha with 80\% of assertiveness.\\

The creation of this service involved the integration of different
tools and techniques, like sox to remove noise and segment audios,
MFC (Mel-Frequency Cepstrum) to extract features from the audio
and SVM to train models. This service was developed in Go with
integration with externals tools and one developed in Python
(the feature extractor).\\

I also developed a tool (also Go) to generate load and evaluate the
assertiveness of captcha services in a generic way
(both image and audio ones). This tool required multiplexing
of results coming from multiple different goroutines and an
{\href{https://github.com/madlambda/spells}{open source library}}
was developed to help solve that problem. The docs for the multiplexer
can be found {\href{https://godoc.org/github.com/madlambda/spells}{here}}.

}
\end{entrylist}

\begin{entrylist}
%------------------------------------------------
\entry
{2015--2017}
{Neoway}
{Florianópolis, Santa Catarina}
{\emph{Lead Software Engineer - Data Capture Team} \\

My main job was to lead the development of a new
data capture architecture that would scale better than
the current one, both in performance as in development
effort required to develop new web scrappers and maintain
the current ones. \\

The work started by substituting the current web scrapper
language (a XML) with Python using the Scrappy framework.
This enabled us to tackle new challenges like parsing
PDF or other complex binary formats (the legacy XML based
scrapper lacked this capability). Another small Python
framework was also developed to aid the development of scrappers
that used Selenium when required. Both frameworks generated
metrics using the StatsD protocol. \\

The next step was to extract logic that resided on the legacy
web scrapper engine into separate services,
like proxy, captcha, and raw storage services.
These services where developed in
Go and exported metrics using the StatsD protocol. Another service
was also developed to integrate the new web scrappers into the
old data pipeline (while the new data pipeline was being developed by
another team). This experience involved a lot of designing and
coding in Go and Python and also brainstorming with other teams
on how to define protocols integrating the data pipeline.\\

Efforts were made to build an automated development environment
that enabled developers to build and debug web scrappers
in isolation of each other. To do this we used Docker and docker-compose.
The deployment and scheduling of web scrappers was solved using
Kubernetes on top of CoreOS at AWS and later at Azure. Kubernetes
was essential since each web scrapper is handled as a independent
unit of deploy and we had more than 200 of them (and also the services).\\

On top of that I was coaching the team on this new architecture
and how to develop the web scrappers in Python, including the
use of automated tests, code reviews and continuous integration
to improve the quality of the web scrappers, solving some serious
issues with bad quality on the captured data.\\

The team was responsible for everything, from coding
to deployment and operations, including maintaining
the Kubernetes cluster healthy. For monitoring we used Sysdig.
I also participated on the screening and hiring process of new
candidates for the team.
}
\end{entrylist}

\begin{entrylist}
%------------------------------------------------
\entry
{2012--2015}
{Dígitro}
{Florianópolis, Santa Catarina}
{\emph{Lead Software Engineer} \\

Worked on the development of a VoIP phone with touchscreen
in a very strict environment
(a Blackfin DSP processor with no MMU and 64MB of RAM).
The project was built from scratch in C and used
DBus to integrate different processes. This project involved
learning considerably about SIP and VoIP state machines
and also memory issues because of the lack of an MMU, like
external memory fragmentation. The development environment
was rather complex and involved cross compilation with
very specific tools. This build of this development environment
was done using Vagrant and Ansible. \\

Some refined functions on the VoIP phone required integration
with our PBX that used a rather complex proprietary protocol.
To solve the integration problem we developed a PBX
REST service in Lua that integrated with the current legacy
protocols exposing them in a more web friendly way. This service
was implemented in Lua. \\

My last project at Dígitro was a solution to web audio playback
with very specific audio effects (like silence removal,
change in pitch) that had to be developed using HTML5.
It was the rewrite of an old system that worked using Flash.
Some of the components of the old project, like Gstreamer
plugins, were reused, but the service itself had to be rewriten
using REST and HTTP to do the media streaming (and control).
This was developed in Javascript (NodeJS) integrating with the
C code for streaming using IPC (Inter Process Communication).
The original audio files where encoded in IMA ADPCM and another
proprietary CODEC and were exported through SFTP. Part of the
problem was also integrating with this legacy infrastructure and
transcode it to a more web friendly format. \\

During these projects I always practiced and advocated TDD
(even on embedded systems in C) and helped people on
the team to write automated tests for their code. I also
coached two new members in the team.
}
\end{entrylist}

\begin{entrylist}
%------------------------------------------------
\entry
{2010--2012}
{Dígitro}
{Florianópolis, Santa Catarina}
{\emph{Software Engineer} \\

I started working on a solution to web audio playback
with very specific audio effects (like silence removal,
change in pitch) that had to be developed using Flash.
To solve that problem I worked with
two different open source C++ projects that did reverse
engineering of the RTMP protocol to develop our own
Flash Media Server. I worked direcly with the integration
of the server playback logic with Gstreamer and the plugins
that enabled the desired effects on playback. \\

The next project was a solution to biometric identification.
This has been solved by building a service around a third party
C library that built and scored voice models.
I developed a REST service in Lua that binded to the C code
that built the voice models and used MongoDB to store the
voice models and perform searches on the database. \\

Then I developed a service to perform searches for specific
words in audio files. The service would replace a old one so
it had to implement the same textual/proprietary stateful protocol
of the old one. The service itself was implemented in Lua
with bindings to C code that did the actual search of
words. \\

}
\end{entrylist}
%------------------------------------------------
\begin{entrylist}
\entry
{2008--2010}
{Dígitro}
{Florianópolis, Santa Catarina}
{\emph{Trainee} \\

I started helping in the development of an
cross platform (Windows and Linux) audio streaming
library for a VoIP softphone, aiming at porting
the current application that was Windows only to Linux.
The library was C code being cross compiled to Windows using mingw.\\

After that I got involved developing a prototype of a voice
biometrics system. It involved building a small web server
in Python that binded to third party libraries that did
the build and scored voice models and a cross platform
Python application that provided a user friendly GUI
to the web server.
}
\end{entrylist}
%------------------------------------------------
\begin{entrylist}
\entry
{2007-2008}
{Cyclops / LAPIX}
{Florianópolis, Santa Catarina}
{\emph{Trainee} \\

Worked on adding new features on the system responsible to
integrate medical equipment to the DICOM system. I developed a
cross platform application that generated a customized GUI
based on a XML description of a form, making it easier to
people working at hospitals to manipulate the forms and save
them back at XML. Like a domain specific XML editor. \\

This involved learning C++ and XML parsing, together with
developing cross platform GUI applications, on this case
using WxWidgets. The code has been tested using CppUnit. \\

I also automated the setup of the development environment
using Python.
}
\end{entrylist}
%------------------------------------------------

%----------------------------------------------------------------------------------------
%	PROJECTS SECTION
%----------------------------------------------------------------------------------------

\pagebreak
\section{open source projects}

\begin{entrylist}
\entry
{2017-now}
{mdtoc}
{\href{https://github.com/madlambda/mdtoc}{https://github.com/madlambda/mdtoc}}
{
A very simple table of contents generator for markdown.
}
%------------------------------------------------
\end{entrylist}

\begin{entrylist}
\entry
{2016-2018}
{nash}
{\href{https://github.com/NeowayLabs/nash}{https://github.com/NeowayLabs/nash}}
{
Nash is a shell language focused on simplicity and having a nicer syntax
than traditional shells and support to containers. It also strives to be
safer than traditional shells.
}
%------------------------------------------------
\end{entrylist}

\begin{entrylist}
\entry
{2016-2018}
{klb}
{\href{https://github.com/NeowayLabs/klb}{https://github.com/NeowayLabs/klb}}
{
klb is used to automate infrastructure creation on AWS and Azure.
I got involved on designing the support for Azure since this was
the tool used to migrate Neoway production infrastructure from
AWS to Azure.
}
%------------------------------------------------
\end{entrylist}

\begin{entrylist}
\entry
{2013}
{CppUTest}
{\href{http://cpputest.github.io}{http://cpputest.github.io}}
{
CppUTest is a C /C++ based unit xUnit test framework for unit
testing and for test-driving code.

In this project I worked both on improving the documentation and
at adding new native types to the mock framework (which involved
some refactoring).
}
%------------------------------------------------
\end{entrylist}

\begin{entrylist}
\entry
{2012}
{GStreamer}
{\href{http://www.gstreamer.net}{http://www.gstreamer.net}}
{
GStreamer is a library for constructing graphs of
media-handling components. I contributed with a plugin
named \emph{removesilence} and some documentation for the
GstCheck documentation.
}
%------------------------------------------------
\end{entrylist}

\begin{entrylist}
\entry
{2010-2011}
{Pattern detection on H.264}
{\href{https://github.com/katcipis/h264.pattern.detection}{https://github.com/katcipis/h264.pattern.detection}}
{
This is my Bachelor's Thesis and it consists of a prototype
of a H.264 CODEC that uses OpenCV and H.264 internal
algorithms to do pattern detection and
object tracking integrated on the encoding process.

Metadata generated on the encoding process is integrated on
the video bitstream on conformance with the standard.
}
%------------------------------------------------
\end{entrylist}

%------------------------------------------------

\begin{entrylist}
\entry
{2010-2011}
{LuaSofia}
{\href{https://github.com/ppizarro/luasofia}{https://github.com/ppizarro/luasofia}}
{
Lua binding for the Sofia-SIP library.
Contributed to the project from the start,
helping to make decisions about the design of the
software and documenting it.
}
%------------------------------------------------
\end{entrylist}

%------------------------------------------------
\begin{entrylist}
\entry
{2010}
{GPS tracking system}
{\href{https://github.com/katcipis/gps.tracking}{https://github.com/katcipis/gps.tracking}}
{
System designed to provide the location of a device
at the receive of a position request using SMS.
}
\end{entrylist}

%------------------------------------------------
\begin{entrylist}
\entry
{2010}
{LuaNotify}
{\href{https://github.com/katcipis/luanotify}{https://github.com/katcipis/luanotify}}
{
Lua library that implements a simple Pub/Sub system
inspired on glib GSignal API.
}
%------------------------------------------------
\end{entrylist}
\pagebreak

\section{presentations}

%------------------------------------------------
\begin{entrylist}
\entry
{2018}
{Object Orientation in Go}
{\href{http://www.thedevelopersconference.com.br/tdc/2018/florianopolis/trilha-golang}{The Developers Conference}}
{

For people that come from a background on Java or other classic object oriented languages
(like C++) there is also some discussion on if Go is actually object oriented.

In this presentation I try to present Go as a language that is more object oriented than these
classic languages, at least according to the original foundations of object orientation.

Presentation source can be found \href{https://github.com/katcipis/my.presentations/blob/master/gooo/gooo.slide}{here}.

}
%------------------------------------------------
\end{entrylist}

%------------------------------------------------
\begin{entrylist}
\entry
{2016}
{Building Resilient Services in Go}
{\href{https://2016.gopherconbr.org/en/}{GopherCon Brazil}}
{

Resilience is not about never failing, but how do you recover from it.
How can you prevent your services from locking down or exhausting all its resources ?
How to perform graceful service degradation ? Can this kind of behaviour be tested properly ?\\

On Go we have some new features, like Contexts, that helps us to model timeouts and cancellation properly.\\

They can be combined with other useful features as select and channels to model timeouts and resource pools,
which can be essential to provide proper service degradation instead of total failure of the system.\\

On this talk I try to answer this questions using new features available on Go 1.7, direct from production ready software.\\

Presentation source can be found \href{https://github.com/katcipis/my.presentations/tree/master/resilient-services-in-go}{here}.

}
%------------------------------------------------
\end{entrylist}

%------------------------------------------------
\begin{entrylist}
\entry
{2016}
{Real Life Kubernetes}
{\href{http://www.thedevelopersconference.com.br/tdc/2016/florianopolis/trilha-devops}{The Developers Conference}}
{

On this presentation we will give a short introduction on Kubernetes and show the experience of learning
and using Kubernetes on production for two very distinct systems.\\

The first one is a data acquisition system, involving running multiple instances of different crawlers,
storage services, captcha breaking services, message brokers (like RabbitMQ) and database integration outside the cluster.\\

The second one is a web application, involving network analysis using graphs with the ultimate goal of fraud prevention.
The application is strongly bounded with the microservices architecture and the twelve factor app.\\

Graph and document databases, cache layers, a message broker and a distributed filesystem are some of
the technologies surrounding the application ecosystem.\\

Presentation source can be found \href{https://github.com/katcipis/my.presentations/tree/master/real-life-kubernetes}{here}.

}
%------------------------------------------------
\end{entrylist}

\end{document}
