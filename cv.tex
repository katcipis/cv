%%%%%%%%%%%%%%%%%%%%%%%%%%%%%%%%%%%%%%%%%
% Friggeri Resume/CV
% XeLaTeX Template
% Version 1.0 (5/5/13)
%
% This template has been downloaded from:
% http://www.LaTeXTemplates.com
%
% Original author:
% Adrien Friggeri (adrien@friggeri.net)
% https://github.com/afriggeri/CV
%
% License:
% CC BY-NC-SA 3.0 (http://creativecommons.org/licenses/by-nc-sa/3.0/)
%
% Important notes:
% This template needs to be compiled with XeLaTeX and the bibliography, if used,
% needs to be compiled with biber rather than bibtex.
%
%%%%%%%%%%%%%%%%%%%%%%%%%%%%%%%%%%%%%%%%%

\documentclass[]{friggeri-cv} % Add 'print' as an option into the square bracket to remove colors from this template for printing

\begin{document}

\header{Tiago}{Katcipis}{Software Engineer} % Your name and current job title/field

%----------------------------------------------------------------------------------------
%	SIDEBAR SECTION
%----------------------------------------------------------------------------------------

\begin{aside} % In the aside, each new line forces a line break
\section{Contact}
Berlin,Germany
+39 349 859 0291
\href{mailto:tiagokatcipis@gmail.com}{tiagokatcipis@gmail.com}
\href{https://github.com/katcipis}{GitHub}
\href{http://www.linkedin.com/pub/tiago-katcipis/1b/273/8b0}{LinkedIn}
\href{http://katcipis.github.io/}{Blog}
\section{Education}
Bachelor of Computer Science, UFSC, Brazil.
\section{Languages}
English, Portuguese.
\section{Skills}
Software architecture and design.
Critical thinking and problem decomposition.
Clear communication, written and spoken.
\section{Programming Languages}
Go, Python, C, C++, Lua, Bash, Javascript.
\section{Protocols}
HTTP, gRPC, SIP, RTP, AMQP.
\section{Automation and Infrastructure}
Kubernetes, AWS, Azure, Google Cloud, Docker, Terraform,
Ansible, Make, Vagrant, Docker Compose.
\section{Monitoring}
Prometheus, Grafana, StatsD, Sysdig.
\end{aside}

%----------------------------------------------------------------------------------------
%	INTRO SECTION
%----------------------------------------------------------------------------------------

\section{Introduction}

\begin{entrylist}
%------------------------------------------------
%------------------------------------------------
\entry
{}
{}
{}
{
I'm a curious programmer who likes to explore different ways to design, build and test software
always trying to understand as much as I can from the entire environment I'm working in.
That led me to do a lot of different things in my 10 years career, from embedded
software in C to distributed systems in a variety of languages and protocols.
I'm passionate about automation and minimalism when building scalable, efficient and flexible software.
}
%------------------------------------------------
\end{entrylist}

%----------------------------------------------------------------------------------------
%	WORK EXPERIENCE SECTION
%----------------------------------------------------------------------------------------

\section{Experience}

\begin{entrylist}
%------------------------------------------------
\entry
{2019--present}
{FromAtoB}
{Berlin, Germany}
{\emph{Software Engineer - Search2Book Team} \\

\begin{itemize}
  \item Implemented new location service from scratch (Go, HTTP, gRPC).
  \item Added PSD2 compliance on payment method storage service (Go).
  \item Helped migrate core services from legacy environment to new GKE (Google Kubernetes Engine) cluster.
  \item Built backup routines for critical service (Google Memorystore).
  \item Improved development environments making them more consistent.
  \item Integral part of the hiring process doing interviews and pair programming sessions.
\end{itemize}

}
\end{entrylist}

\begin{entrylist}
%------------------------------------------------
\entry
{2017--2019}
{Neoway}
{Florianópolis, Brazil}
{\emph{Software Engineer - Data Platform Team} \\

\begin{itemize}
  \item Led the migration of the entire data platform from AWS to Azure.
  \item Developed tools to automate building infrastructure, like {\href{https://github.com/NeowayLabs/klb}{klb}}.
  \item Created new service to solve audio captchas (Go, Python, SVM).
  \item Prototyped image captcha solver using TensorFlow.
  \item Did 3 different presentations in 2 different conferences.
\end{itemize}
}
\end{entrylist}

\begin{entrylist}
%------------------------------------------------
\entry
{2015--2017}
{Neoway}
{Florianópolis, Brazil}
{\emph{Lead Software Engineer - Data Capture Team} \\

\begin{itemize}
  \item Led development of a new data capture architecture.
  \item Implementation of multiple services for the new architecture (Go).
  \item Implementation of a web scraping framework used to build more than 100 scrapers (Python).
  \item Built automated monitoring system with domain specific metrics (Sysdig, StatsD).
  \item Coached the team on better testing practices and TDD.
  \item Fully automated dev environments and deployment (Docker, Docker Compose).
  \item First team in the company to deploy and use Kubernetes to manage more than 100 deployments.
\end{itemize}
}
\end{entrylist}

\begin{entrylist}
%------------------------------------------------
\entry
{2010--2015}
{Dígitro}
{Florianópolis, Brazil}
{\emph{Software Engineer} \\

\begin{itemize}
  \item Developed VoIP phone with color touchscreen from scratch (C on a Blackfin DSP).
  \item Automated development environment for cross compilation (Ansible, Vagrant).
  \item Replaced legacy audio service that used Flash (RTMP) with an HTTP/HTML5 solution (NodeJS,C).
  \item Built new REST service to integrate with company PBX solution (proprietary protocol).
  \item Built a customized audio playback system (Flash, RTMP, GStreamer, C++).
  \item Built a biometric identification service (HTTP, Lua, C, MongoDB).
  \item Coached team on automated testing and TDD.
\end{itemize}
}
\end{entrylist}

%------------------------------------------------
\begin{entrylist}
\entry
{2008--2010}
{Dígitro}
{Florianópolis, Brazil}
{\emph{Trainee} \\

\begin{itemize}
  \item Worked on making Windows only VoIP softphone cross platform (C, GStreamer, RTP, SIP).
  \item Prototype of biometric identification service (Python, GTK, C).
  \item Prototype of face detection system with processing on the edges (C, OpenCV).
\end{itemize}

}
\end{entrylist}
%------------------------------------------------

\section{Projects}

\begin{entrylist}
\entry
{}
{mdtoc}
{\href{https://github.com/madlambda/mdtoc}{https://github.com/madlambda/mdtoc}}
{
A very simple table of contents generator for markdown.
}
%------------------------------------------------
\end{entrylist}

\begin{entrylist}
\entry
{}
{nash}
{\href{https://github.com/NeowayLabs/nash}{https://github.com/NeowayLabs/nash}}
{
Nash is a shell language focused on simplicity and having a nicer syntax
than traditional shells and support to containers. It also strives to be
safer than traditional shells.
}
%------------------------------------------------
\end{entrylist}

\begin{entrylist}
\entry
{}
{klb}
{\href{https://github.com/NeowayLabs/klb}{https://github.com/NeowayLabs/klb}}
{
klb is used to automate infrastructure creation on AWS and Azure.
}
%------------------------------------------------
\end{entrylist}

\section{Presentations}

%------------------------------------------------
\begin{entrylist}
\entry
{2018}
{Object Orientation in Go}
{\href{http://www.thedevelopersconference.com.br/tdc/2018/florianopolis/trilha-golang}{The Developers Conference}}
{
Presented the Go object model as something closer to the original
idea from Alan Kay then classic object oriented languages like Java and C++.
Presentation source can be found \href{https://github.com/katcipis/my.presentations/blob/master/gooo/gooo.slide}{here}.
}
%------------------------------------------------
\end{entrylist}

%------------------------------------------------
\begin{entrylist}
\entry
{2016}
{Building Resilient Services in Go}
{\href{https://2016.gopherconbr.org/en/}{GopherCon Brazil}}
{
Presented new features on Go, like Contexts, that helps to model timeouts and
cancellation properly, which are essential to build a resilient system.\\

Presentation source can be found \href{https://github.com/katcipis/my.presentations/tree/master/resilient-services-in-go}{here}.
}
%------------------------------------------------
\end{entrylist}

%------------------------------------------------
\begin{entrylist}
\entry
{2016}
{Real Life Kubernetes}
{\href{http://www.thedevelopersconference.com.br/tdc/2016/florianopolis/trilha-devops}{The Developers Conference}}
{
On this presentation we will give a short introduction on Kubernetes and show the experience of learning
and using Kubernetes on production.

Presentation source can be found \href{https://github.com/katcipis/my.presentations/tree/master/real-life-kubernetes}{here}.
}
%------------------------------------------------
\end{entrylist}

%------------------------------------------------

\end{document}
