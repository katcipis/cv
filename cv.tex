%%%%%%%%%%%%%%%%%%%%%%%%%%%%%%%%%%%%%%%%%
% Friggeri Resume/CV
% XeLaTeX Template
% Version 1.0 (5/5/13)
%
% This template has been downloaded from:
% http://www.LaTeXTemplates.com
%
% Original author:
% Adrien Friggeri (adrien@friggeri.net)
% https://github.com/afriggeri/CV
%
% License:
% CC BY-NC-SA 3.0 (http://creativecommons.org/licenses/by-nc-sa/3.0/)
%
% Important notes:
% This template needs to be compiled with XeLaTeX and the bibliography, if used,
% needs to be compiled with biber rather than bibtex.
%
%%%%%%%%%%%%%%%%%%%%%%%%%%%%%%%%%%%%%%%%%

\documentclass[]{friggeri-cv} % Add 'print' as an option into the square bracket to remove colors from this template for printing

\begin{document}

\header{tiago}{katcipis}{software engineer} % Your name and current job title/field

%----------------------------------------------------------------------------------------
%	SIDEBAR SECTION
%----------------------------------------------------------------------------------------

\begin{aside} % In the aside, each new line forces a line break
\section{contact}
+39 349 859 0291
\href{mailto:tiagokatcipis@gmail.com}{tiagokatcipis@gmail.com}
\href{https://github.com/katcipis}{GitHub}
\href{http://www.linkedin.com/pub/tiago-katcipis/1b/273/8b0}{LinkedIn}
\href{http://katcipis.github.io/}{Blog}
\section{languages skill}
English: advanced, Portuguese: advanced
\section{programming languages}
Go, C, C++, Python, Lua, Bash, Javascript, Nash
\section{protocols}
HTTP, gRPC, SIP, RTP, AMQP
\section{cloud}
Kubernetes, AWS, Azure, Google Cloud, Docker, Terraform
\section{automation}
Ansible, Make
\section{dev environment}
Vagrant, Docker Compose
\section{monitoring}
Prometheus, Grafana, StatsD, Sysdig
\end{aside}

%----------------------------------------------------------------------------------------
%	EDUCATION SECTION
%----------------------------------------------------------------------------------------

\section{education}

\begin{entrylist}
%------------------------------------------------
%------------------------------------------------
\entry
{2006--2011}
{Bachelor of Computer Science}
{UFSC}
{}
%------------------------------------------------
\end{entrylist}

%----------------------------------------------------------------------------------------
%	WORK EXPERIENCE SECTION
%----------------------------------------------------------------------------------------

\section{experience}

\begin{entrylist}
%------------------------------------------------
\entry
{2019--now}
{FromAtoB}
{Berlin, Germany}
{\emph{Software Engineer - Data Platform Team} \\

Led the migration of the entire data platform from AWS to Azure,
which resulted in a fully automated infrastructure.
To build this automation an open source project was developed
at Neoway to deploy infrastructure at Azure called
{\href{https://github.com/NeowayLabs/klb}{klb}} which is
a stateless tool to build infrastructure.\\

After the infrastructure migration I helped to solve an
audio captcha, something that had not
been done before at Neoway.Doing that involved some researching
on current techniques and culminated on a service capable
of solving a specific audio captcha with 80\% of assertiveness.\\

The creation of this service involved the integration of different
tools and techniques, like sox to remove noise and segment audios,
MFC (Mel-Frequency Cepstrum) to extract features from the audio
and SVM to train models. This service (and helper tooling) was developed in Go
with integration to some open source Python tools for feature extraction.\\
}
\end{entrylist}

\begin{entrylist}
%------------------------------------------------
\entry
{2017--2019}
{Neoway}
{Florianópolis, Brazil}
{\emph{Software Engineer - Data Platform Team} \\

Led the migration of the entire data platform from AWS to Azure,
which resulted in a fully automated infrastructure.
To build this automation an open source project was developed
at Neoway to deploy infrastructure at Azure called
{\href{https://github.com/NeowayLabs/klb}{klb}} which is
a stateless tool to build infrastructure.\\

After the infrastructure migration I helped to solve an
audio captcha, something that had not
been done before at Neoway.Doing that involved some researching
on current techniques and culminated on a service capable
of solving a specific audio captcha with 80\% of assertiveness.\\

The creation of this service involved the integration of different
tools and techniques, like sox to remove noise and segment audios,
MFC (Mel-Frequency Cepstrum) to extract features from the audio
and SVM to train models. This service (and helper tooling) was developed in Go
with integration to some open source Python tools for feature extraction.\\
}
\end{entrylist}

\begin{entrylist}
%------------------------------------------------
\entry
{2015--2017}
{Neoway}
{Florianópolis, Brazil}
{\emph{Lead Software Engineer - Data Capture Team} \\

My main job was to lead the development of a new
data capture architecture that would scale better than
the current one, both in performance as in development
effort required to develop new web scrappers and maintain
the current ones. It was also an objective to have
improved visibility on the health of services
and the whole data capture process, together with
improved code quality.\\

The work started by substituting the current web scrapper
language (XML) with Python using the Scrappy framework.
This enabled us to tackle new challenges like parsing
PDF or other complex binary formats and also enabled better
testing and the introduction of continuous integration.

I also helped to define and implement services with clear
boundaries like proxy, captcha, and raw storage services.
These services where developed in Go and exported metrics using
the StatsD protocol.\\

An automated development environment was created using
Docker and docker-compose and
the deployment and scheduling of more than 200 web scrappers
in production was automated using Kubernetes.
}
\end{entrylist}

\begin{entrylist}
%------------------------------------------------
\entry
{2012--2015}
{Dígitro}
{Florianópolis, Brazil}
{\emph{Lead Software Engineer} \\

Worked on the development of a VoIP phone with touchscreen
in a very strict environment
(a Blackfin DSP processor with no MMU and 64MB of RAM).
The project was built from scratch in C and used
DBus to integrate different processes. The development environment
was rather complex and involved cross compilation with
very specific tools. This build of this development environment
was done using Vagrant and Ansible. \\

Also worked on REST services. One to a new and simpler API to
the company PBX, which was written in Lua. And another one
was a solution to web audio playback with very specific audio effects
(like silence removal, change in pitch, DTMF/Fax detection)
that was a reimplementation
of a legacy system that used RTMP, but now using HTTP.

This was developed in Javascript (NodeJS) integrating with the
C code for transcoding of different media formats to a format
more friendly to browsers.\\

During these projects I always practiced and advocated TDD
(even on embedded systems in C) and helped people on
the team to write automated tests for their code.
}
\end{entrylist}

\begin{entrylist}
%------------------------------------------------
\entry
{2010--2012}
{Dígitro}
{Florianópolis, Brazil}
{\emph{Software Engineer} \\

I started working on a solution to web audio playback
with very specific audio effects (like silence removal,
change in pitch) that had to be developed using Flash (RTMP).
To solve that problem I worked with
two different open source C++ projects that did reverse
engineering of the RTMP protocol to develop our own
Flash Media Server. I worked directly with the integration
of the server playback logic with Gstreamer and the plugins
that enabled the desired effects on playback. \\

The next project was a solution to biometric identification
using a third party C library that built and scored voice models.
I developed a REST service in Lua that integrated with C code
that built the voice models and used MongoDB to store the
voice models and perform searches on the database. \\
}
\end{entrylist}
%------------------------------------------------
\begin{entrylist}
\entry
{2008--2010}
{Dígitro}
{Florianópolis, Brazil}
{\emph{Trainee} \\

Helped in the development of an
cross platform (Windows and Linux) audio streaming
library for a VoIP softphone, aiming at porting
the current application that was Windows only to Linux. I also
got involved in the development of a prototype for a voice
biometrics system.
}
\end{entrylist}
%------------------------------------------------
\begin{entrylist}
\entry
{2007-2008}
{Cyclops / LAPIX}
{Florianópolis, Brazil}
{\emph{Trainee} \\

Worked on adding new features on the system responsible to
integrate medical equipment to the DICOM system, developing a
cross platform domain specific graphical XML editor.
This involved learning C++ and XML parsing, together with
developing cross platform GUI applications, on this case
using WxWidgets. The code has been tested using CppUnit.
}
\end{entrylist}
%------------------------------------------------

%----------------------------------------------------------------------------------------
%	PROJECTS SECTION
%----------------------------------------------------------------------------------------

\pagebreak
\section{open source projects}

\begin{entrylist}
\entry
{2017-now}
{mdtoc}
{\href{https://github.com/madlambda/mdtoc}{https://github.com/madlambda/mdtoc}}
{
A very simple table of contents generator for markdown.
}
%------------------------------------------------
\end{entrylist}

\begin{entrylist}
\entry
{2016-2018}
{nash}
{\href{https://github.com/NeowayLabs/nash}{https://github.com/NeowayLabs/nash}}
{
Nash is a shell language focused on simplicity and having a nicer syntax
than traditional shells and support to containers. It also strives to be
safer than traditional shells.
}
%------------------------------------------------
\end{entrylist}

\begin{entrylist}
\entry
{2016-2018}
{klb}
{\href{https://github.com/NeowayLabs/klb}{https://github.com/NeowayLabs/klb}}
{
klb is used to automate infrastructure creation on AWS and Azure.
I got involved on designing the support for Azure since this was
the tool used to migrate Neoway production infrastructure from
AWS to Azure.
}
%------------------------------------------------
\end{entrylist}

\begin{entrylist}
\entry
{2013}
{CppUTest}
{\href{http://cpputest.github.io}{http://cpputest.github.io}}
{
CppUTest is a C /C++ based unit xUnit test framework for unit
testing and for test-driving code.

In this project I worked both on improving the documentation and
at adding new native types to the mock framework (which involved
some refactoring).
}
%------------------------------------------------
\end{entrylist}

\begin{entrylist}
\entry
{2012}
{GStreamer}
{\href{http://www.gstreamer.net}{http://www.gstreamer.net}}
{
GStreamer is a library for constructing graphs of
media-handling components. I contributed with a plugin
named \emph{removesilence} and some documentation for the
GstCheck documentation.
}
%------------------------------------------------
\end{entrylist}

\begin{entrylist}
\entry
{2010-2011}
{Pattern detection on H.264}
{\href{https://github.com/katcipis/h264.pattern.detection}{https://github.com/katcipis/h264.pattern.detection}}
{
This is my Bachelor's Thesis and it consists of a prototype
of a H.264 CODEC that uses OpenCV and H.264 internal
algorithms to do pattern detection and
object tracking integrated on the encoding process.

Metadata generated on the encoding process is integrated on
the video bitstream on conformance with the standard.
}
%------------------------------------------------
\end{entrylist}

%------------------------------------------------

\begin{entrylist}
\entry
{2010-2011}
{LuaSofia}
{\href{https://github.com/ppizarro/luasofia}{https://github.com/ppizarro/luasofia}}
{
Lua binding for the Sofia-SIP library.
Contributed to the project from the start,
helping to make decisions about the design of the
software and documenting it.
}
%------------------------------------------------
\end{entrylist}

%------------------------------------------------
\begin{entrylist}
\entry
{2010}
{GPS tracking system}
{\href{https://github.com/katcipis/gps.tracking}{https://github.com/katcipis/gps.tracking}}
{
System designed to provide the location of a device
at the receive of a position request using SMS.
}
\end{entrylist}

%------------------------------------------------
\begin{entrylist}
\entry
{2010}
{LuaNotify}
{\href{https://github.com/katcipis/luanotify}{https://github.com/katcipis/luanotify}}
{
Lua library that implements a simple Pub/Sub system
inspired on glib GSignal API.
}
%------------------------------------------------
\end{entrylist}
\pagebreak

\section{presentations}

%------------------------------------------------
\begin{entrylist}
\entry
{2018}
{Object Orientation in Go}
{\href{http://www.thedevelopersconference.com.br/tdc/2018/florianopolis/trilha-golang}{The Developers Conference}}
{

For people that come from a background on Java or other classic object oriented languages
(like C++) there is also some discussion on if Go is actually object oriented.

In this presentation I try to present Go as a language that is more object oriented than these
classic languages, at least according to the original foundations of object orientation.

Presentation source can be found \href{https://github.com/katcipis/my.presentations/blob/master/gooo/gooo.slide}{here}.

}
%------------------------------------------------
\end{entrylist}

%------------------------------------------------
\begin{entrylist}
\entry
{2016}
{Building Resilient Services in Go}
{\href{https://2016.gopherconbr.org/en/}{GopherCon Brazil}}
{

Resilience is not about never failing, but how do you recover from it.
How can you prevent your services from locking down or exhausting all its resources ?
How to perform graceful service degradation ? Can this kind of behaviour be tested properly ?\\

On Go we have some new features, like Contexts, that helps us to model timeouts and cancellation properly.\\

They can be combined with other useful features as select and channels to model timeouts and resource pools,
which can be essential to provide proper service degradation instead of total failure of the system.\\

On this talk I try to answer this questions using new features available on Go 1.7, direct from production ready software.\\

Presentation source can be found \href{https://github.com/katcipis/my.presentations/tree/master/resilient-services-in-go}{here}.

}
%------------------------------------------------
\end{entrylist}

%------------------------------------------------
\begin{entrylist}
\entry
{2016}
{Real Life Kubernetes}
{\href{http://www.thedevelopersconference.com.br/tdc/2016/florianopolis/trilha-devops}{The Developers Conference}}
{

On this presentation we will give a short introduction on Kubernetes and show the experience of learning
and using Kubernetes on production for two very distinct systems.\\

The first one is a data acquisition system, involving running multiple instances of different crawlers,
storage services, captcha breaking services, message brokers (like RabbitMQ) and database integration outside the cluster.\\

The second one is a web application, involving network analysis using graphs with the ultimate goal of fraud prevention.
The application is strongly bounded with the microservices architecture and the twelve factor app.\\

Graph and document databases, cache layers, a message broker and a distributed filesystem are some of
the technologies surrounding the application ecosystem.\\

Presentation source can be found \href{https://github.com/katcipis/my.presentations/tree/master/real-life-kubernetes}{here}.

}
%------------------------------------------------
\end{entrylist}

\end{document}
