%%%%%%%%%%%%%%%%%%%%%%%%%%%%%%%%%%%%%%%%%
% Friggeri Resume/CV
% XeLaTeX Template
% Version 1.0 (5/5/13)
%
% This template has been downloaded from:
% http://www.LaTeXTemplates.com
%
% Original author:
% Adrien Friggeri (adrien@friggeri.net)
% https://github.com/afriggeri/CV
%
% License:
% CC BY-NC-SA 3.0 (http://creativecommons.org/licenses/by-nc-sa/3.0/)
%
% Important notes:
% This template needs to be compiled with XeLaTeX and the bibliography, if used,
% needs to be compiled with biber rather than bibtex.
%
%%%%%%%%%%%%%%%%%%%%%%%%%%%%%%%%%%%%%%%%%

\documentclass[]{friggeri-cv} % Add 'print' as an option into the square bracket to remove colors from this template for printing

\begin{document}

\header{tiago}{katcipis}{software engineer} % Your name and current job title/field

%----------------------------------------------------------------------------------------
%	SIDEBAR SECTION
%----------------------------------------------------------------------------------------

\begin{aside} % In the aside, each new line forces a line break
\section{contact}
+(55)(48)991060132
~
\href{mailto:tiagokatcipis@gmail.com}{tiagokatcipis@gmail.com}
\href{https://github.com/katcipis}{GitHub}
\href{http://www.linkedin.com/pub/tiago-katcipis/1b/273/8b0}{LinkedIn}
\href{http://katcipis.github.io/}{Blog}
\section{languages skill}
English: advanced (reading, writing); intermediate (speaking)
\section{programming languages}
C, C++, Go, Python, Lua, Nash, Bash, Javascript
\section{protocols}
DBus, RTMP, HTTP, SIP, RTP, AMQP
\section{cloud}
Kubernetes, AWS, Azure, Docker, CoreOS, Terraform
\section{documentation}
Gtk-Doc, Doxygen, Docbook, LDoc, LaTeX, Docco
\section{automation}
Ansible, Make
\section{dev environment}
Vagrant, Docker Compose
\section{monitoring}
Prometheus, Grafana, StatsD, Sysdig
\end{aside}

%----------------------------------------------------------------------------------------
%	EDUCATION SECTION
%----------------------------------------------------------------------------------------

\section{education}

\begin{entrylist}
%------------------------------------------------
%------------------------------------------------
\entry
{2006--2011}
{Bachelor of Computer Science}
{UFSC}
{}
%------------------------------------------------
\end{entrylist}

%----------------------------------------------------------------------------------------
%	WORK EXPERIENCE SECTION
%----------------------------------------------------------------------------------------

\section{experience}

\begin{entrylist}
%------------------------------------------------
\entry
{2018--2019}
{Neoway}
{Florianópolis, Santa Catarina}
{\emph{Software Engineer - Lambda Team} \\

    As a software engineer on the lambda team I was responsible for researching
    and developing new captcha solvers (audio and image).
    Solving audio captchas involved studying and applying the following
    algorithms and tools:

\begin{itemize}
    \item Noise removal using sox
    \item Audio segmentation by silence using sox
    \item Extracting features (MFCC) from segmented audio
    \item Normalize extracted features with rescaling
    \item Train SVM model
    \item Recognize characters using trained SVM model
    \item Label image captchas using TensorFlow
\end{itemize}

Main accomplishments:\\

\begin{itemize}
    \item Development of tool to measure assertiveness of captcha solving services
    \item Solved audio captcha with 80\% of assertiveness
    \item Solved image captcha with 40\% assertiveness using TensorFlow
    \item Tooling to automate dataset and trained models upload at Azure BlobFS
\end{itemize}


}
\end{entrylist}


\begin{entrylist}
%------------------------------------------------
\entry
{2017--2018}
{Neoway}
{Florianópolis, Santa Catarina}
{\emph{Software Engineer - Data Platform Team} \\

    As a software developer on the data platform team I started to be
    responsible for a more broad context involving multiple teams, building not
    only services but also tooling to be used by those teams. \\
    Besides the technical challenges of providing a platform I also performed
    code reviews and coaching across multiple teams.
    There was a lot of challenges like:\\

\begin{itemize}
    \item Migrating the entire infrastructure of multiple teams from AWS to Azure
    \item Defining network topology and security measures
    \item Creating backups for VMs with different operational systems at Azure
    \item Specify a new DSL to develop web scrappers
    \item Scale the ingestion of huge CSV files (terabytes per file)
    \item Define solutions for infrastructure building that crossed multiple teams boundaries
\end{itemize}

Main accomplishments:

\begin{itemize}
    \item Fully automated infrastructure provisioning at Azure
    \item Easy to build parallel version of the entire infrastructure (including networking and security)
    \item Development and testing of a library to automate infrastructure building
    \item Specified and developed generic backup strategy at Azure
    \item Entire infrastructure access through a bastion host as a security measure
    \item New service that allowed arbitrarily huge files to be ingested easily (scaling horizontaly)
\end{itemize}
}
\end{entrylist}

\begin{entrylist}
%------------------------------------------------
\entry
{2015--2017}
{Neoway}
{Florianópolis, Santa Catarina}
{\emph{Software Engineer - Data Capture Team} \\

    Technical lead of the team responsible for capturing data from the web and publishing it to the entire company,
which involved solving hard problems as:\\

\begin{itemize}
    \item Scraping the web (Scrapy and Selenium)
    \item Parsing data from multiple formats (HTML, PDF, SWF)
    \item Maintaining all downloaded raw data at S3
    \item Defining and documenting good protocols for proper service integration
    \item Developing services on different languages, like Go and Python
    \item Migrating the whole data pipeline to a new service oriented architecture
    \item Coach the team on good test techniques for the new architecture
    \item Developed new services to provide good proxies and captcha breaking
\end{itemize}

Besides the technical challenges I also helped the team to apply some
development practices, like TDD (Test Driven Development),code review
and continuous integration.
Together we built a DevOps culture to enable infrastructure as code
on our team, we where responsible for the whole solution, from the development
to testing and deployment (including monitoring the production system).\\

Main accomplishments:

\begin{itemize}
    \item Technical leadership and coaching for a team of 6 people
    \item Built a new crawling framework, from scratch to production
    \item Fully automated development environment with docker compose
    \item Fully automated dashboard and alert creation on Sysdig Cloud
    \item Cluster orchestration at AWS/Azure using Kubernetes, CoreOS and Docker
    \item Migrating all infrastructure from Codero to AWS
    \item Later, migrated the infrastructure from AWS to Azure
    \item Active participation on building the tools to automate infrastructure on Azure
    \item Aided on the migration of more than 100 crawlers to Kubernetes
    \item Actively participated on the screening and interviews of new candidates for the team
    \item Coached new members
    \item Implemented a real time monitoring for the data production pipeline using StatsD and Sysdig
    \item Gave talks inside the company and on events like TDC (The Developers Conference) and GopherCon
\end{itemize}
}
\end{entrylist}

\begin{entrylist}
%------------------------------------------------
\entry
{2012--2015}
{Dígitro}
{Florianópolis, Santa Catarina}
{\emph{Lead Software Engineer} \\

Worked on a VoIP phone with touchscreen
developing in a very strict environment
(a Blackfin DSP processor with no MMU and 64MB of RAM).
The project was built from scratch in C and used
DBus to integrate different processes. This project involved
learning considerably about SIP and VoIP state machines
and also memory issues because of the lack of an MMU, like
external memory fragmentation. The development environment
was rather complex and involved cross compilation with
very specific tools. This build of this development environment
was done using Vagrant and Ansible. \\

Some refined functions on the VoIP phone required integration
with the our PBX, to solve that problem we developed a PBX
REST service in Lua that integrated with the current legacy
protocols.\\

}
\end{entrylist}

\begin{entrylist}
%------------------------------------------------
\entry
{2010--2012}
{Dígitro}
{Florianópolis, Santa Catarina}
{\emph{Software Engineer} \\

I started working on a solution to web audio playback
with very specific audio effects (like silence removal,
change in pitch) that had to be developed using Flash.
To solve that problem I worked with
two different open source C++ projects that did reverse
engineering of the RTMP protocol to develop our own
Flash Media Server. I worked direcly with the integration
of the server playback logic with Gstreamer and the plugins
that enabled the desired effects on playback. \\

The next project was a solution to biometric identification.
This has been solved by building a service around a third party
C library that built and scored voice models.
I developed a REST service in Lua that binded to the C code
that built the voice models and used MongoDB to store the
voice models and perform searches on the database. \\

Then I developed a service to perform searches for specific
words in audio files. The service would replace a old one so
it had to implement the same textual/proprietary stateful protocol
of the old one. The service itself was implemented in Lua
with bindings to C code that did the actual search of
words. \\

}
\end{entrylist}
%------------------------------------------------
\begin{entrylist}
\entry
{2008--2010}
{Dígitro}
{Florianópolis, Santa Catarina}
{\emph{Trainee} \\

I started helping in the development of an
cross platform (Windows and Linux) audio streaming
library for a VoIP softphone, aiming at porting
the current application that was Windows only to Linux.
The library was C code being cross compiled to Windows using mingw.\\

After that I got involved developing a prototype of a voice
biometrics system. It involved building a small web server
in Python that binded to third party libraries that did
the build and scored voice models and a cross platform
Python application that provided a user friendly GUI
to the web server.
}
\end{entrylist}
%------------------------------------------------
\begin{entrylist}
\entry
{2007-2008}
{Cyclops / LAPIX}
{Florianópolis, Santa Catarina}
{\emph{Trainee} \\

Worked on adding new features on the system responsible to
integrate medical equipment to the DICOM system. I developed a
cross platform application that generated a customized GUI
based on a XML description of a form, making it easier to
people working at hospitals to manipulate the forms and save
them back at XML. Like a domain specific XML editor. \\

This involved learning C++ and XML parsing, together with
developing cross platform GUI applications, on this case
using WxWidgets. The code has been tested using CppUnit. \\

I also automated the setup of the development environment
using Python.
}
\end{entrylist}
%------------------------------------------------

%----------------------------------------------------------------------------------------
%	PROJECTS SECTION
%----------------------------------------------------------------------------------------

\pagebreak
\section{open source projects}

\begin{entrylist}
\entry
{2017-now}
{mdtoc}
{\href{https://github.com/madlambda/mdtoc}{https://github.com/madlambda/mdtoc}}
{
A very simple table of contents generator for markdown.
}
%------------------------------------------------
\end{entrylist}

\begin{entrylist}
\entry
{2016-2018}
{nash}
{\href{https://github.com/NeowayLabs/nash}{https://github.com/NeowayLabs/nash}}
{
Nash is a shell language focused on simplicity and having a nicer syntax
than traditional shells and support to containers. It also strives to be
safer than traditional shells.
}
%------------------------------------------------
\end{entrylist}

\begin{entrylist}
\entry
{2016-2018}
{klb}
{\href{https://github.com/NeowayLabs/klb}{https://github.com/NeowayLabs/klb}}
{
klb is used to automate infrastructure creation on AWS and Azure.
I got involved on designing the support for Azure since this was
the tool used to migrate Neoway production infrastructure from
AWS to Azure.
}
%------------------------------------------------
\end{entrylist}

\begin{entrylist}
\entry
{2013}
{CppUTest}
{\href{http://cpputest.github.io}{http://cpputest.github.io}}
{
CppUTest is a C /C++ based unit xUnit test framework for unit
testing and for test-driving code.

In this project I worked both on improving the documentation and
at adding new native types to the mock framework (which involved
some refactoring).
}
%------------------------------------------------
\end{entrylist}

\begin{entrylist}
\entry
{2012}
{GStreamer}
{\href{http://www.gstreamer.net}{http://www.gstreamer.net}}
{
GStreamer is a library for constructing graphs of
media-handling components. I contributed with a plugin
named \emph{removesilence} and some documentation for the
GstCheck documentation.
}
%------------------------------------------------
\end{entrylist}

\begin{entrylist}
\entry
{2010-2011}
{Pattern detection on H.264}
{\href{https://github.com/katcipis/h264.pattern.detection}{https://github.com/katcipis/h264.pattern.detection}}
{
This is my Bachelor's Thesis and it consists of a prototype
of a H.264 CODEC that uses OpenCV and H.264 internal
algorithms to do pattern detection and
object tracking integrated on the encoding process.

Metadata generated on the encoding process is integrated on
the video bitstream on conformance with the standard.
}
%------------------------------------------------
\end{entrylist}

%------------------------------------------------

\begin{entrylist}
\entry
{2010-2011}
{LuaSofia}
{\href{https://github.com/ppizarro/luasofia}{https://github.com/ppizarro/luasofia}}
{
Lua binding for the Sofia-SIP library.
Contributed to the project from the start,
helping to make decisions about the design of the
software and documenting it.
}
%------------------------------------------------
\end{entrylist}

%------------------------------------------------
\begin{entrylist}
\entry
{2010}
{GPS tracking system}
{\href{https://github.com/katcipis/gps.tracking}{https://github.com/katcipis/gps.tracking}}
{
System designed to provide the location of a device
at the receive of a position request using SMS.
}
\end{entrylist}

%------------------------------------------------
\begin{entrylist}
\entry
{2010}
{LuaNotify}
{\href{https://github.com/katcipis/luanotify}{https://github.com/katcipis/luanotify}}
{
Lua library that implements a simple Pub/Sub system
inspired on glib GSignal API.
}
%------------------------------------------------
\end{entrylist}
\pagebreak

\section{presentations}

%------------------------------------------------
\begin{entrylist}
\entry
{2018}
{Object Orientation in Go}
{\href{http://www.thedevelopersconference.com.br/tdc/2018/florianopolis/trilha-golang}{The Developers Conference}}
{

For people that come from a background on Java or other classic object oriented languages
(like C++) there is also some discussion on if Go is actually object oriented.

In this presentation I try to present Go as a language that is more object oriented than these
classic languages, at least according to the original foundations of object orientation.

Presentation source can be found \href{https://github.com/katcipis/my.presentations/blob/master/gooo/gooo.slide}{here}.

}
%------------------------------------------------
\end{entrylist}

%------------------------------------------------
\begin{entrylist}
\entry
{2016}
{Building Resilient Services in Go}
{\href{https://2016.gopherconbr.org/en/}{GopherCon Brazil}}
{

Resilience is not about never failing, but how do you recover from it.
How can you prevent your services from locking down or exhausting all its resources ?
How to perform graceful service degradation ? Can this kind of behaviour be tested properly ?\\

On Go we have some new features, like Contexts, that helps us to model timeouts and cancellation properly.\\

They can be combined with other useful features as select and channels to model timeouts and resource pools,
which can be essential to provide proper service degradation instead of total failure of the system.\\

On this talk I try to answer this questions using new features available on Go 1.7, direct from production ready software.\\

Presentation source can be found \href{https://github.com/katcipis/my.presentations/tree/master/resilient-services-in-go}{here}.

}
%------------------------------------------------
\end{entrylist}

%------------------------------------------------
\begin{entrylist}
\entry
{2016}
{Real Life Kubernetes}
{\href{http://www.thedevelopersconference.com.br/tdc/2016/florianopolis/trilha-devops}{The Developers Conference}}
{

On this presentation we will give a short introduction on Kubernetes and show the experience of learning
and using Kubernetes on production for two very distinct systems.\\

The first one is a data acquisition system, involving running multiple instances of different crawlers,
storage services, captcha breaking services, message brokers (like RabbitMQ) and database integration outside the cluster.\\

The second one is a web application, involving network analysis using graphs with the ultimate goal of fraud prevention.
The application is strongly bounded with the microservices architecture and the twelve factor app.\\

Graph and document databases, cache layers, a message broker and a distributed filesystem are some of
the technologies surrounding the application ecosystem.\\

Presentation source can be found \href{https://github.com/katcipis/my.presentations/tree/master/real-life-kubernetes}{here}.

}
%------------------------------------------------
\end{entrylist}

\end{document}
